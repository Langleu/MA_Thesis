
\input{layout.tex}

% Front Matter
\singlespacing
\newcommand{\titleLineOne}{Graph-based Recommendation of Optimized IT-Service Deployment Scripts based on crowdsourced Data}
\newcommand{\titleLineTwo}{}
\newcommand{\titleLineThree}{}
\newcommand{\documentdate}{\today}
\newcommand{\studentname}{Lars Lange}
\newcommand{\abstracttextde}{Infrastructure as Code hat in den letzten Jahren durch die DevOps-Bewegung viel an Bedeutung gewonnen, aber bisher hat noch niemand ein Empfehlungssystem für Deployment-Skripte entwickelt, das basierend auf benutzerdefinierten Eingaben ähnliche Skripte vorschlägt, die durch einen vorberechneten Wert ermittelt werden. Dieser Wert umfasst „Best Practices“, Schwachstellen, Dateilänge und die Frage, ob das Skript ausführbar ist.
Die Implementierung eines verteilten Crawlers, einer verteilten Analyse und eines Empfehlungssystems werden im Detail beschrieben, die durch die Verwendung des Strategiemusters zukunftssicher sind.
Der Hauptfokus lag auf Deployment-Skripten rund um Docker-Compose und es wurden ca. 127.000 Benutzer, 160.000 Repositorien und 139.000 Docker-Compose-Skripte vom System gecrawlt und verarbeitet, was insgesamt 409.000 Dienste und Images ergab.
Von diesen 139.000 Skripten wurden insgesamt 18.000 erfolgreich analysiert, indem die verteilte Analyse genutzt wurde, die eine automatisierte Pipeline in einer isolierten Umgebung durchführte, um Konsistenz zu gewährleisten.
Die Auswertung der Daten aus zwei verschiedenen Blickwinkeln, datenbezogen und grafisch, offenbarte einen beliebten Verwendungszweck von Docker-Compose-Skripten, gab aber auch implizite Einblicke in die Community rund um das Skript, speziell auf GitHub.
}
\newcommand{\abstracttext}{Infrastructure as Code has gained a lot of traction in recent years due to the DevOps movement, but so far no one has covered a recommendation system for deployment scripts yet, which suggests those based on user-defined input to recommend similar scripts established through a pre-calculated score. This score covers best practices, vulnerabilities, file length, and whether the script was executable.
Furthermore, implementation of a distributed crawler, distributed analysis, and recommender system are described in detail, which are future proof by utilizing the strategy pattern.
The main focus relied on deployment scripts around Docker-Compose and approximately 127,000 users, 160,000 repositories, and 139,000 Docker-Compose scripts were crawled and processed by the system, which resulted in a total of 409,000 services and images.
Out of these 139,000 scripts, a total of 18,000 were successfully analysed by utilizing the distributed analysis, which ran an automated pipeline in an isolated environment to guarantee consistency.
Evaluating the data from two different perspectives, data related and graph related, revealed a popular usage point of Docker-Compose scripts, but also gave implicit insights on the community around it, specifically on GitHub.
}
%\newcommand{\acktext}{This chapter is optional. First of all, I would like to...}

\begin{document} 

\input{frontmatter.tex}

% Body Matter (use input to add chapters)
\onehalfspacing

\chapter{Introduction}
All the reader needs to know to get introduced to the topic. Motivate, state the problem and give a hint to your contribution. What is this thesis about? Why is it interesting? Give the reader a brief idea of the structure of the thesis. One to three pages.

\chapter{Background}
\label{sec:background}
The background is divided into 5 sections, which are \hyperref[sec:background-iac]{Infrastructure as Code}, \hyperref[sec:background-kubernetes]{Kubernetes}, \hyperref[sec:background-graph]{Graph database}, \hyperref[sec:background-automation]{Automation Software}, and \hyperref[sec:background-recommender]{Recommender System}. These sections shall convey the necessary topics to further understand the contribution.

\section{Infrastructure as Code}
\label{sec:background-iac}
Infrastructure as Code describes the concept of defining infrastructure-related elements, like servers, networks, databases, or applications in source code. This brings along various advantages like versioning since it is manifested in a file and can be kept in version control \cite{iacArmon}. Automation tools can be built around the concept of Infrastructure as Code and allow integration with Continuous Delivery and Container Orchestrations, granting the advantage of immutable infrastructure since elements are rebuilt instead of updating running applications. Tools that integrate the concept of Infrastructure as Code are Ansible, Helm, Chef, Puppet, Kustomize, Docker-Compose, and many more. This thesis not only adopts the concept of Infrastructure as Code, but also analyses and assess those according to a score, in this particular case for Docker-Compose, and recommends similar manifests using a generalized knowledge graph.

\section{Kubernetes}
\label{sec:background-kubernetes}
Kubernetes, according to their documentation \cite{whatKubernetes} is an open-source platform for managing containerized workloads and services. It utilizes declarative manifests for configuring workloads and, thereby, embeds into the idea of Infrastructure as Code. Furthermore, it can be said that Kubernetes is an orchestration platform and was initially published by Google as an open-source project in 2014 \cite{kubernetes2014}\cite{googleContrib}. Compared to traditional deployments, Kubernetes offers additional isolation due to the usage of container runtimes like Docker or containerd and allows scaling applications across multiple servers without much configuration.

Containerization is the concept of isolating environments for single applications within an OS. This allows to package the software with all of its dependencies and run it consistently on any underlying infrastructure \cite{containerization}, as long as it supports the same architecture. It is often used with the concept of a microservice architecture since it allows running all services isolated from each other and enables scalability. Kubernetes makes use of containerization as previously mentioned by supporting container runtimes like Docker or containerd.

Kubernetes offers the concept of a Pod, which is the smallest deployable unit in Kubernetes \cite{pods}. A pod unites one or more containers into a single unit, which all share the same underlying storage and network resources. This has the advantage, that one could combine a multitude of tools into one pod, making it easier to process certain data without the need for additional storage logic. Example use cases are Continuous Integration, where one could bundle all required tools into one pod or simply a frontend application, which requires the build of static resources.

Another concept that pods are used for is the so-called init containers. The concept of an initialization container is to run before the actual container starts and allows to interact with the underlying storage as well \cite{init}. In the example of the frontend application, the init container could build those static resources before the actual application even starts. An init container can not be used to seed a database with data since the database would only start if the init container has finished. This is important since the database used in this thesis does not have out of the box support for declarative seeding of the database and requires additional tooling around it, explained further in the section Open Source Contributions \ref{sec:opensource}. Additionally it is used for the declarative Jenkins setup to download all required plugins before the actual Jenkins instance starts as it requires the plugins before start up.

\section{Graph database}
\label{sec:background-graph}
A graph database is a relational database but treats relationships between data as equally important to the data itself \cite{graphdb}. The data storage depends on the implementation of the graph database but is usually not done in tables.
A Knowledge Graph is based on a graph database with additional decision support usually utilizing an AI\footnote{Artificial intelligence} or Machine Learning \cite{knowledgegraph}. While this thesis could have been implemented by using a traditional relational database, the advantages of using a graph database, or more precisely a Knowledge Graph, are that implicit observations can be done, due to the additional focus on edges. A knowledge Graph is not that different from a normal graph, but focuses more on semantically rich graph data \cite{graknKnowledge}. Therefore, the nodes describe entities and the edges are relations of those nodes. Another point is the visual analysis of such data, which is quite important when it comes to social networks, which in the case of GitHub is partly given.
In the case of this thesis, the Knowledge Graph and Graph database "Grakn.ai"\footnote{https://grakn.ai} was chosen, due to its intuitive query language called Graql and support for a declarative configuration of the database model.

\section{Automation Software}
\label{sec:background-automation}
This thesis requires the execution of possibly several thousands tests, while this could be done manually it would be quite an error-prone operation. Therefore, automation software is required to guarantee consistent execution of the required tests. Some Infrastructure as Code tools themselves, are automation software as well e.g. Chef and Puppet, but with a focus on infrastructure, while this approach focused more on a generic execution of a workflow.
Jenkins is an open-source automation server \cite{jenkins}, which due to its plugin supports can be used as Continuous Integration or Continuous Delivery system as well. It is extensible and allows itself to be moulded according to one's requirements. Using the right plugins it can use Kubernetes to orchestrate its workloads across a cluster, making it distributed.
A workflow engine is an application, which orchestrates tasks according to a workflow or business process and manages those tasks \cite{workflow}. It can be used for fully automated processes like CI/CD or any other automation. Integration with such things as Kubernetes depends on the implementation of the chosen workflow engine.
Both Jenkins and a workflow engine would have been sufficient for this thesis since both are capable of executing a workflow, but Jenkins due to its closer integration with Kubernetes already brings proper workload isolation.

\section{Recommender System}
\label{sec:background-recommender}
A recommender system is a system, which suggests items, which could be of interest for a user. For this, the system collects preferences of the user and uses those to find a suitable item \cite{recommender}.
The recommendation of deployment scripts based on user-defined input has not been done yet and is an interesting topic since it requires the classification of manifests according to a numerical value. It shall support Infrastructure as Code related work fields, e.g. DevOps, by suggesting deployment scripts, which are close to their requirement and give an inspiration of how to properly define one themselves by using the indicated numerical value, or score.

Recommender systems are categorized in one of fours types, which are Collaborative Filtering, Content-based, Knowledge-based and, a Hybrid approach \cite{recommender}\cite{recommender2}.
A Collaborative Filtering system recommends items that have been preferred by other users with related choices \cite{recommender}.
Content-based recommends items that are similar to items that the user has preferred in the past \cite{recommender}.
Knowledge-based recommender systems use user-defined constraints or cases to recommend items based on the given information \cite{recommender2}.
A hybrid system can use both Collaborative Filtering and Content-based recommendations \cite{recommender}\cite{recommender2}.

Since this thesis relies on the user-defined constraints it is a knowledge-based recommender system. Other systems could possibly be implemented as well by building a community around the outcome of this thesis.


\chapter{Contribution}

Most important chapter of the thesis. Describes what the author contributes as research. Discusses intuition, motivation, describes and reasons about necessity of proposed elements. Defines theses based on reasonable assumptions. Discusses relevant aspects of contribution. Approximately 30 to 40 pages. Can be split into multiple chapters.

\section{Microservice Architecture}
\subsection{Distributed Crawler}
\subsubsection{Information funnel}
\subsubsection{Orchestration}
\subsubsection{Crawling}
\todo{partly covers GitHub crawling}
\todo{GitHub Rate limiting}
\subsubsection{Processing}
\subsubsection{Unified base}
\todo{meant as the same code base for client/server - interchangeable}
\subsection{Distributed Analysis}
\subsubsection{Choice of Tooling}
\todo{e.g. Jenkins + Kubernetes, conftest}
\subsubsection{Importance of Isolation}
\todo{e.g. CI and dind or kind}
\subsubsection{Vulnerability Scanning}
\subsubsection{Calculation of Score}
\subsubsection{Orchestrator / Funnel}
\todo{Normalization}
\todo{CVSS}
\subsection{Database - Grakn.ai}
\todo{Deduplication}
\subsubsection{Entity relationship model}
\subsubsection{Open Source Contributions}
\todo{e.g. own tooling for creating gexf compatible graphs - Limitations of provided tools (electron)}
\todo{e.g. own tooling for seeding the db model}
\subsection{Frontend}
\subsubsection{Content-based Recommendation}
\section{Expandability}
\todo{current focus on docker-compose, but can easily be extended to further sources, may it be crawling/processing/analysing/showing}
\section{IaC of implementation}


\chapter{Evaluation}

This chapter is divided into the evaluation of the quantitative metrics captured by the implementation of this thesis and the network analysis of the graph, which can be generated out of the graph database and further analysed.

All previously described implementations were wrapped into individual docker containers, to first and foremost adopt "Infrastructure as Code", but also to use Kubernetes as scaleable infrastructure. The setup was running on a self hosted Kubernetes cluster using 5 baremetal machines, consisting of 2 master nodes and a total of 5 worker nodes, including the master nodes.

The graph database was occupying one of the 5 worker nodes, using a special taint to only schedule the database on this single node. Due to the usage of baremetal servers, and extra schedule taints, the local disk was directly mounted into the container to provide faster throughput compared to using a virtual disk. Each individual server provided 8 cores and 16 Gigabyte of memory. Typical cloud offers from Azure, AWS or GCP were not accessible.

The remaining implementations, e.g. Jenkins, the crawler and the proxy were all scheduled across the remaining 4 nodes.

The byte range of 0 bytes to 384.000 bytes were crawled within two weeks and no additional crawling was done afterwards, meaning the results are purely from one run, which covered all possible bytes that GitHub allows. The analysis took another two weeks and was done after the two weeks of crawling the results to not overload the graph database. The reasoning for this will be explained further down below.

Considering the crawler implementation one can compare the total amount of available "docker-compose" files that the GitHub API provides to the total amount the crawlers have crawled. As already described in the contribution about the crawler \todo{add reference to chapter}, the GitHub API has some limitations that affect the implementation as well. While providing a horizontally scaleable solution, which is only limited by the amount of GitHub tokens supplied, the GitHub API still has some nowhere described complications. This thesis and included test execution were all done under the terms of the GitHub Terms of Services, which states that the crawling of public data is only allowed for scientific works and require the results to be public \todo{quote TOS}. The TOS also states that the usage of the GitHub API is only allowed with a fair usage in mind, meaning one is not allowed to differ too much of the average usage of certain GitHub API routes. Not complying with those rule will likely result in a termination of ones account. During the data collection period, a total of three GitHub accounts were created with the sole purpose of crawling public data, which is only available with a valid GitHub account. Due to the excessive usage of the code search GitHub API route one out of three accounts received a so-called shadow ban. A shadow ban is a system to hide the fact that the user received a temporary termination. A shadow banned user can still use every GitHub API, but the results returned by the API are always empty even though they still conform the expected schema. GitHub as a provider of this free service does not provide any additional documentation about this system. It is not known at which point the account received the shadow ban, but it is to assume that this limitation was applied after the crawling period of two weeks, since the crawler still reported valid results back compared to two weeks later.

Besides this limitation the only other limiting factor was the 16 Gigabyte of memory for a graph database, which turned out to be not an issue by the memory itself, but the graph database had some blocking operations in the write process, which was heavily relied on. In a recent version this issue was fixed. This issue might have had an impact on the total amount of crawled files, since the database started to block any further insertions. To detect those issues a service called Sentry\footnote{https://sentry.io} was used, which sent alerts for not handled issues like the blocking of the database. This might have had an implication on the total amount of crawled files. The community edition of Grakn does not provide any Kubernetes manifests, thereby, manifests and settings have to be created by oneself, which likely results in a not entirely production ready setup, since the production ready setup with multi sharding is only reserved for the commercial usage.

In the following the actual amount of crawled entries will be compared to the possible and total amount of crawled items that the GitHub API returns. Both will be compared to each other by using a column chart.

\begin{figure}[H]
    \centering
    \includegraphics[scale=0.5]{graphics/stats_total.png}
    \caption{Chart representing the total amount of possible results using the GitHub API for the term "docker-compose"}
    \label{fig:stats_total}
\end{figure}

The chart \ref{fig:stats_total} represents the total amount of possible results for the term "docker-compose" using only the GitHub API. The Y-Axis represents the total amount of found occurrences and the X-Axis represents bytes. For this graphical representation the range from 0 to 175.000 was used, since between 175 Kilobyte and 384 Kilobyte are only an additional 200 results. There is a total of 1,1 million results in the chart, most of them clustered around the file size of 100 to 200 bytes, which can be explained due to "docker-compose" files being rather simple and short compared to other deployment scripts. 

\begin{figure}[H]
    \centering
    \includegraphics[scale=0.5]{graphics/stats_range.png}
    \caption{Cutout from \ref{fig:stats_total} for the range from 0 to 3000 bytes }
    \label{fig:stats_range}
\end{figure}

The cutout from the range 0 to 3000 bytes as seen in figure \ref{fig:stats_range} shows that the most files are in a range from 100 to 400 bytes and then slowly traverses towards 0, with as seen in figure \ref{fig:stats_total} occasional occurrences in a higher byte range. Both charts \ref{fig:stats_total} and \ref{fig:stats_range} represent the best case if GitHub would return more than a 1000 results for a single query and are both incomplete as well, since GitHub will only return the amount of found results till it receives a timeout. Meaning that running the same query twice would likely result in different results, as the API receives a timeout and returns all found results so far. This behaviour occurs as well when querying single pages, since GitHub only allows a maximal result of 100 entries. This could mean that for running one query on page 10 one could receive a 100 results and for another one close to 0, since the API received a timeout before. In case of this thesis this issue was dismissed, as the author has no influence on the actual implementation of GitHub and it could be mitigated by running the crawling process multiple times for the same bytes.

\begin{figure}[H]
    \centering
    \includegraphics[scale=0.5]{graphics/stats_range_max_possible.png}
    \caption{Cutout from \ref{fig:stats_total} for the range from 0 to 3000 bytes, but a maximum of 1000 results }
    \label{fig:stats_max_possible}
\end{figure}

The chart in figure \ref{fig:stats_max_possible} shall represent the maximum amount of possible results, that can be crawled, since the GitHub API only returns a maximum of 1000 results for a single query. As explained prior for the chart \ref{fig:stats_range}, this does not mean that one will receive a 1000 results, since the GitHub API is nondeterministic.

Further on, the charts for the actual crawled data will be shown, which resulted in 140.000 valid deployments. Valid, since only parseable deployment files were inserted into the database, see \todo{add reference to the explanation in contribution} in the chapter about the implementation.

\begin{figure}[H]
    \centering
    \includegraphics[scale=0.5]{graphics/deployment_stats_total.png}
    \caption{Chart representing the total amount of possible results using the crawler implementation for the term "docker-compose"}
    \label{fig:deployment_total}
\end{figure}

Comparing the figure \ref{fig:stats_total} and the figure \ref{fig:deployment_total}, one may notice a similar representation, which validates the overall implementation, since clusters for similar bytes were successfully crawled with occasional results in the higher byte area.
The chart \ref{fig:deployment_total} also shows that the maximum crawleable number of a 1000 results was never reached, which could be explained by either the nondeterministic behavior of the GitHub API or the blocking process of the database. On the other hand this could simply be explained by non parseable deployment scripts as well, since a lot of people have especially in the lower byte range scripts that are the bare minimum or contain syntactical issues, which cause the parsing to fail and at the same time render the file useless for the "docker-compose" executable as well, since it can not parse the file either.

\begin{figure}[H]
    \centering
    \includegraphics[scale=0.5]{graphics/deployment_stats_range.png}
    \caption{Cutout from \ref{fig:deployment_total} for the range from 0 to 3000 bytes}
    \label{fig:deployment_range}
\end{figure}

The chart \ref{fig:deployment_range} still resembles the chart \ref{fig:stats_range}, but it is visible that that the previously described shadow ban has occurred earlier than expected, since the chart \ref{fig:deployment_range} shows multiple spots where barely any results were returned, e.g. 161-201, 441-481 or 521-641, which mirror the system of a shadow ban. The implementation featured crawler windows, which consisted of a byte range that were individually assigned to each crawler, which in this case could have been assigned to the shadow banned account. During the execution period it was not visible that the account was affected, since it acted as expected and returned results according to the schema.

Overall the GitHub API returned a total amount of 940.000 possible crawleable entries, which can still differ in numbers when actually crawled for due to the way the GitHub API works. A total of 140.000 valid deployment scripts were inserted into the database, which represents 15\% of the total crawleable amount. Limiting factors were in this case the nondeterministic behavior of the GitHub API, the shadow ban system of GitHub and the blocking operations of the graph database. Except the nondeterministic behavior of the GitHub API all other issues can be mitigated for future work, by providing more valid GitHub tokens, which can be switched in case of shadow banned accounts and a more non-blocking graph database. Grakn already provided a patch to resolve the blocking issue, but in the non community edition the graph database can be scaled across multiple Kubernetes nodes as well, since the community edition does not provide any Kubernetes manifests and relied on the implementation of the author.

In total the graph database contains 4,2 million entries, which can be divided into roughly a million nodes, a million edges and two million attributes. Which can be further split down as follows:

\begin{table}[h!]
    \centering
    \begin{tabular}{ |c|c| }
    \hline
    Type & Amount \\
    \hline
         count & 4.279.664 \\
         attributes & 2.546.547 \\
         \hline
         nodes & 836.337 \\
         \hline
         user & 127.084 \\
         repository & 160.506\\
         deployment & 139.469\\
         service & 409.277\\
         \hline
         edges & 896.780 \\
         \hline
         own & 159.834 \\
         contain & 194.797 \\
         include & 404.736 \\
         depend & 137.412 \\
    \hline
    \end{tabular}
    \caption{Graph Database total counts}
    \label{graph_database_total_counts}
\end{table}

%Second most important chapter. Verifies the theses defined in the previous chapter. Tries to evaluate and analyze the contribution in qualitative or quantitative terms. Ends with a discussion. Approximately 20 to 30 pages. Can be split into multiple chapters.

% Hier solltest du die limitations deiner Contirbution versuchen zu analysieren. Dies können quantitative sachen sein, wie die limitationen des crawlers wie oft gecrawlet werden kann über github api und wie oft dann andere ergebnisse rauskommen. andere sachen wären zB ob der von dir erstellte score "qualitativ gut" IaC bewerten kann. Für diese unterschiedlichen Punkte sollten folgende Unterpunkte folgen:
%a) Evaluation Setup: hier wird der versuchsaufbau und die durchführung beschrieben. welche größen werden zB gemessen sollte auch beschrieben werden. Womit wird vielleicht das ergebnis verglichen?
%b) Results: Darstellung der Ergebnisse und Beschreibung der Ergebnisse
%c) Discussion: Bewerte die Ergebnisse und ordnete sie ein.
%Am Ende des Kapitels sollte dann eine summary kommen, was der leser gelernt haben soll und welche fragen noch offen geblieben sind für future work und welche forschungsrichtungen man diese arbeit weiter entwickeln kann.

\section{Graph Analysis}
\section{TBD - General Analysis}

\chapter{State of the Art}
\label{sec:stateofart}
The current state of the art can be divided into 4 categories, which are Crawling of Code \ref{sec:coc}, Infrastructure as Code Analysis \ref{sec:iaca}, and Recommender Systems for Developers \ref{sec:rsfd}, and Code Analysis \ref{sec:codeAnalysis}. Those categories are chosen since there is no direct paper related to this thesis, but papers in each of those categories, which are within their categories related to this thesis.
% Related work. Present state of research and applied solutions concerning the different aspects relevant to the thesis. Discuss differences and similarities to other solutions to the given tackled problem. Approximately 10 to 15 pages.

% Länger als Background. State of the Art ~ 4-5 Seiten, aber wichtig Background immer weniger
% (0,5-1 Seite je paper), die sehr ähnlich sind zu deiner Arbeit. Weniger genau Paper, die teilaspekte deiner Arbeit beschreiben (ca. 0,5 Seite je paper). Zusammenfassned dann alle paper die perifär ähnliches beschreiben (typisch sind hier so 4-8 paper pro halbe seite.

\section{Crawling of Code}
\label{sec:coc}
In regards to "Crawling of Code", one of the papers is "Mining the Network of the Programmers: A Data-Driven Analysis of GitHub" \cite{mining} published by Ma et al., which has the goal of collecting public data of GitHub and analysing it as a social network. Within their work, they implemented a distributed crawler, which collected more than 2 million user data from the profile pages on GitHub. Their implementation used a scheduler connected to a MySQL database, which recorded the data, and worker nodes, which used a Breadth-First Search algorithm to crawl the users' profile pages on GitHub by using the follower and followings lists. They applied machine learning to the collected data by selecting certain features like the total amount of stars the user received or the number of repositories the user owns and visualized their findings in charts.
Compared to this thesis they took the approach of a distributed crawler as well since it covers more data more quickly, but they had the advantage of only having to access publicly available data from the GitHub page without having to use their API and its limitations. Due to the usage of MySQL, their findings lack the analysis of social network-related metrics like the average shortest path, the clustering coefficient and its possibility of a small world problem, and many more, which could have been derived by using a graph or graph database. Using a BFS algorithm compared to GitHub API to find possible repositories that own a "docker-compose" file is not feasible, since there are more than 40 million public repositories\cite{githubpublicrepos}, which could potentially contain such files. Crawling each repository would require one to crawl through each possible directory or once more use the GitHub API, this would likely result in an IP address blockage since a GitHub account is not required. In regards to the code collection, this thesis is the most related one in terms of the approach, since a distributed crawler was used.

The paper "Influence analysis of Github repositories" \cite{influencegithub} by Hu et al. deals with the analysis of important and influential GitHub repositories based on publicly available data. Using this data, they created a graph based on the stars a repository received over time by utilizing the GitHub star event. They used several sources to collect this data. One of those sources was the GitHub API to gather metadata on users and repositories. Another source is the "GH Archive"\footnote{https://www.gharchive.org/}, which offers all available event data from 2011 to 2020 and was used as the main driver of their data collection since GitHub itself only offers up to 300 events via their API with a maximum age of 90 days. The setup didn't require any distributed crawler logic, as the "GH Archive" offered all available data without the need to crawl data oneself. In direct comparison, the solution using the "GH Archive" as a data source is a good starting point, but it only offers event data. There is no public repository containing a global index of all available files that are kept in public GitHub repositories. The difference in the approach relies on the usage of a pre-collected data source, which as previously stated is not available for the data, which was required for this thesis. In terms of data storage, a similar approach can be assumed to be used since it was no further described, except for using a database and extracting a graph out of it. While this can be achieved by using a SQL or NoSQL database, the advantages of a graph database overshadow the extra steps required to using a pure SQL or NoSQL database.

A related master thesis "Crawling and Analyzing Repository in GitHub" \cite{Zhang2016CrawlingAA} by Zhongpei Zhang deals with the collection of user and repository data of all repositories with more than 500 stars of GitHub. The goal is to cluster the usage of programming languages according to repositories and users and analyse the data and how programming languages might be coupled together depending on their usage. The thesis does not describe their crawler implementation nor does it describe the database they used, but they describe in detail the process of crawling GitHub, which shows similarities to this thesis in terms of limitations. Zhongpei Zhang heavily relied on the GitHub API in terms of crawling user and repository data, which are both endpoints that allow up to 5,000 queries an hour compared to only 1,800 queries per hour for the code API. Another similarity is the usage of query parameters to create unique queries and receives additional data compared to a standard search.

The complementary use of the GitHub API and a GitHub event collection service seems to be a prevalent case for researchers. The same case for the paper "The quest for open source projects that use UML: mining GitHub" \cite{0.1145/2976767.2976778} by Hebig et al., which looks at the usage of UML in the context of open source projects. For this, the GitHub archive of GHTorrent\footnote{https://ghtorrent.org/} was used, which contains event data from 2012 to 2019, and contemplated with the GitHub API to get additional metadata. The potential problem of using such an archive is outdated data since the repository containing the related files could be private by then or the file has moved to a different location, possibly causing one to for non-existing files and exhausting the GitHub rate limit of 5,000 requests. In their paper, they state that they needed over 2 weeks to successfully crawl all metadata of 1,240,00 UML model files. While they don't state whether multiple GitHub accounts were used for the retrieval, they do mention later that for actual file analysis a total of 21 GitHub accounts were used. Since the file analysis took 6 weeks and the metadata retrieval over 2 weeks it can be assumed that parallelization was not used for the metadata retrieval and at the same time raises the question why 21 GitHub accounts were necessary for cloning public repositories. Overall the paper presented an interesting approach on the data collection of GitHub, which could be partly adopted for this thesis as well but still, a distributed crawler could make sense if launching this thesis as a longterm project since the system is expandability and allows further integration due to its REST API.

\section{Infrastructure as Code Analysis}
\label{sec:iaca}
The first paper in regards to the "Infrastructure as Code Analysis" is called "Cloud WorkBench – Infrastructure-as-Code Based Cloud Benchmarking" \cite{cloudworkbench} by Scheuner et al. and describes the creation of a cloud benchmarking Web service with a focus on Infrastructure-as-a-Service clouds. Since cloud benchmarking can be error-prone if done manually, automation was required to verify the validity of the results. For this, the concept of Infrastructure as Code was used, since it verifies that the same resources are created if the same manifest was used and thereby providing reusability and consistency. The software Chef\footnote{https://www.chef.io/} was used as automation and orchestration system to run the benchmarks, which were previously defined in Infrastructure as Code manifests. Comparing this thesis and the paper, it shows similarities in goals and execution since both deal with the benchmarking of a related field in a consistent and automated way. While the paper deals with the benchmarking of cloud providers, this thesis deals with the benchmarking of Infrastructure as Code related deployment scripts. Both projects relied on the concept of Infrastructure as Code since it provides a reusable and consistent way to execute benchmarks with the usage of automation software, which deals with the execution.

The second paper "Testing Idempotence for Infrastructure as Code" \cite{idempotence} by Hummer et al. deals with the testing of the idempotency of Infrastructure as Code scripts with a focus on Chef. For this, a distributed prototype was implemented, which tested 298 so-called cookbooks provided by the Opscode community. Their system consists of a Web interface, which controls the test execution, and various virtual machines, which are each running a test agent. This test agent is an implementation, that executes, intercepts tests, and inserts the results into a MongoDB. The tests itself are run within a Linux container (LXC), which offer proper environment isolation and allow parallel test execution. Overall the idea of this paper goes in a similar direction as this thesis, just with a different goal. Publicly available data was used, while the paper didn't require an additional crawler, due to the manual selection of deployment scripts according to different criteria. Test execution was done in form of a pipeline and proper workload isolation using the most popular tools at that time, which supports the approach in this thesis. While this paper focuses on the idempotence of deployment scripts it does not create a score in any sense, but rather whether a script behaves in the same way if executing multiple times, which could be used as an additional indicator for this thesis in case of the implementation of a Chef strategy.

An analysis of all available non-forked Dockerfiles was done in October 2016 and is described in the paper "An Empirical Analysis of the Docker Container Ecosystem on GitHub" \cite{empirical} by Cito et al. Instead of crawling public data via the GitHub API, a public GitHub archive was used, which is not further described nor properly explained how. The "GH Archive"\footnote{https://www.gharchive.org/} does not contain any data about the contents of a repository, except for event data. To gain additional metadata about the set of 70,197 Dockerfiles the GitHub API was used to be able to connect the information of the Dockerfile better to such things as programming languages or project size. Their goal was to compare the usage of Docker in popular projects to the general GitHub community. A relational database was used as storage. Their implementation involved a Java application, which applied a linter to each Dockerfile and its revisions and inserted the results into the relational database. Manual execution of 560 repositories was done, which resulted in 34\% not being able to build. The lack of version pinning resulted in linting errors for 28.6\% of all Dockerfiles. This analysis was done in 2016 and similar observations can still be done in 2020 in terms of version pinning. The approach was done in a similar way, but with the focus on Dockerfiles instead of "docker-compose" files, while the paper does not try to calculate a score, but rather just express in a binary way, whether a manifest follows certain community-defined best practices or not. Additionally, a distributed system was not used, which could have eased the manual testing and possibly enabled the testing of 70,000 Dockerfiles. Therefore, the paper solely focuses on an overall analysis of all available Dockerfiles on GitHub, while this thesis focuses on deployment scripts and recommending those according to a calculated score. Ideas of this paper, like linting Dockerfiles, could be included in the future work or utilizing the same system to create strategies to execute all 70,000 Dockerfiles in a consistent way.

There is no known paper that deals with the scoring of Infrastructure as Code related manifests. All of the papers presented deal with one aspect in particular, which could in return be used as an indicator of how good a deployment script is in terms of a final score. Overall the presented papers show similar approaches in terms of workload isolation and automation since those attributes are required to verify consistent results.

\section{Recommender Systems for Developers}
\label{sec:rsfd}
"Knowledge-aware Recommender System for Software Development" \cite{Nguyen2018KnowledgeawareRS} is a paper written by Nguyen et al. and proposes a recommender system with a focus on open-source software by utilizing a knowledge graph. This system shall help developers to find similar projects with the same focus or suggesting libraries that similar projects have used. In total, they implemented all known types of a recommender system, each for a different use case. Similar approaches were taken in regards to the knowledge graph, but besides that, the use cases are not comparable since the paper focuses on Maven\footnote{https://maven.org}, an artifact repository.

The paper "A Declarative Recommender System for Cloud Infrastructure Services Selection" \cite{costcalculatorcloud} by Zhang et al. defines a recommender system for Infrastructure as a Service provider. Its recommendation is based on the cost calculation of user-defined services, which those available providers offer. The system shall help a user to decide which provider to select depending on the use case they have. Their implementation consisted of a Web service using JavaScript frameworks as frontend and Java as backend utilizing a MySQL database. The recommender system can be described as a knowledge recommendation, similar to this thesis since it relies on user-defined input to be able to recommend data. Since the data does not resemble a possible social network, nor have any important relations, a relational database was sufficient. A data crawler was not required as well since the data has to be manually collected from the cloud providers public documentation. A score calculation does not take place, since the Web service will return the cost calculations for all cloud providers and a user has to decide and compare themselves between those. A knowledge-based recommendation was used as it relies on user-defined data, similar to this thesis, but has a shortage in actually recommending a final item since all options are presented to the user, which still require the user to review all items to determine themselves the best candidate.

"Prompter: A Self-confident Recommender System" \cite{6976143} by Ponzanelli et al. describes an IDE plugin called Prompter, that supports developers by automatically checking Stack Overflow discussions based on the context the developer is currently in and suggests those if a certain confidence is reached. It silently observes the context in the IDE and searches the Web and Stack Overflow discussions and evaluates the relevance of the results based on multiple factors. Those factors are code, conceptual, and Stack Overflow community related, which together result in a numerical value, which will trigger a notification if a user set threshold is reached. Thereby, this paper fits not only in the recommender system category but also the code analysis, since it not only described a recommender system but also a statical code analysis. Similar to this thesis their approach is based on crowdsourced data from multiple sources and use a knowledge-based recommendation system, which makes use of a self calculated numerical value established through multiple aspects. While their implementation is packaged as an IDE plugin, this thesis provides a Web service.

\section{Code Analysis}
\label{sec:codeAnalysis}
Code analysis, in particular static code analysis, is a useful method to assure a certain code quality or find errors before even running a program. While this thesis does not analyse source code of programming languages, it does still statically analyse the deployment scripts for best practices and prior to this even for compatibility of the file schema.
"Empirical Analysis of Static Code Metrics for Predicting Risk Scores in Android Applications" \cite{androidAlenezi} by Alenezi and Almomani conducted an empirical study on the impact of static code metrics and their relation to security vulnerabilities in Android applications. They analysed 1407 pre-selected Android applications by statically analysing the code using Sonarqube and used Androrisk to receive a risk score using fuzzy logic. Furthermore, they categorized the risk scores into No, Low, Medium, and High for an easier prediction model. By using the Spearman's Rank Correlation Coefficient they showed a direct correlation of static code violations to a higher risk score and used this combined with machine learning to classify other applications as well based on this data set. Taking this direct correlation into account it can be observed in deployment scripts as well, since the 5 deployment scripts with the highest score, as seen in Table \ref{images_with_highest_score}, were the ones with the least amount of vulnerabilities and following the best practices, which were derived by applying a static code analysis. To conclude a clearer picture an empirical analysis would need to be carried out on the existing crawled data, but the subset presented in Table \ref{images_with_highest_score} does lead to this assumption.

The paper "Context Is King: The Developer Perspective on the Usage of Static Analysis Tools" \cite{8330195} by Vassallo et al. carried out a study on the usage of static analysis tools in different contexts. Research questions covered context usage, configuration, and warning awareness in different contexts. Out of the 42 study participants, 37\% use static analysis tools in Continuous Integration, 33\% in Code Reviews, and 30\% in local programming environments. In terms of configuration 51\% only configure the static analysis tool once. Depending on the context developers inspect different warnings. Code reviews being more on the style and code redundancy site, local programming is more about code structure and logic, and CI is mostly to observe concurrency and handling errors.
This paper showed how deeply static analysis tools are embedded into the development flow of software for different use cases, which can be seen as one of the reasons to use such metrics in the calculation of the final score in this thesis. Those are represented by the static analysis of the best practices and the parsing of the deployment script, which could be summarized as code style checks, whereas the vulnerability scanner is a static analysis of the used images, reporting possible vulnerability and attack surface.

\chapter{Conclusion}
\label{sec:conclusion}
This thesis, crawled approximately 127,000 users, 160,000 repositories, 139,000 "docker-compose" scripts, which covered 409,000 services, and resulting in a total amount of 900,000 edges and 840,000 nodes.
Out of these 139,000 "docker-compose" scripts a total amount of 18,000 scripts were successfully analysed within 2 weeks using an automation system and proper workload isolation.

An evaluation was carried out from two different perspectives, which are the general data analysis and the visual analysis by utilizing a graph.
The general analysis revealed a heavy focus on web development related topics since the most used programming languages are JavaScript and PHP. The top 5 used databases revealed a focus on relational databases compared to non relational databases with a share of 60\% compared to 40\% if only taking the top 5 used databases into account. Besides that the general analysis showed a focus on message queues and blockchain technologies as well.
35\% of all versions described the latest image in the "docker-compose" file, which is discouraged by the factor of reproducibility since it can't be guaranteed that the image is still the same in a week and thereby hindering successful execution.
All of the deployment scripts with a higher score showed the usage of alpine based images or security hardened images, which should be the best practice in terms of security for Docker.

The self-hosting of a production ready Kubernetes cluster showed its limitations when using it in direct relation to Continuous Integration since the etcd storage was filled so quickly that it started blocking the Kubernetes API.

Focusing on the visual analysis using the graph derived from the knowledge graph, revealed that some bigger communities around deployment scripts exist but the majority of 99.9\% are smaller communities with less than a 100 members, and thereby the overall system is rather disconnected and not showing any signs of a small world problem. Additionally the visual analysis revealed business models around GitHub and "docker-compose", by utilizing GitHub as a storage for a "docker-compose" as a service system or the App builder kit, which created individual repositories, containing "docker-compose" files, for each created application. Furthermore implicit metrics can be derived from the graph, which can be useful to determine popular tutorials or popular projects without relying on an explicit rating defined by a subset of users.

An implementation of a future proof crawler, analysis and recommender system was described by the utilization of the strategy pattern.
The scoring system was explained, which is the main driver of the knowledge-based recommender system. Further on, the score is derived by using an automated and isolated pipeline to determine metrics, such as best practices, vulnerability scores, whether it is executable, and file length.

Related work in this field always covered only one aspect of a deployment script and defined a binary score of whether this aspect applied or not. This thesis focused on the creation of a numerical value, which can be utilized further by the recommender system.
Overall the score gives an indication of how good a deployment script is in terms of vulnerabilities and best practices.

Future work could extend on this thesis and implement additional strategies for other deployment scripts. The possibility of a hybrid recommender system exists by creating a community around it and introducing a mix of collaborative filtering and knowledge-based recommendation. Additional meta information could be crawled from the GitHub API to create interlinkages between users by using their followers and followings lists to create possibly bigger communities.

%One page. What have we learned in/through this thesis?

%Expected thesis length: 90 pages (+-10\%)

%\chapter{Notes}

Just a collection of random notes about possible findings/decisions. Will later be converted into actual content.

\section{GitHub Crawling}

Different approaches:\\
web ui vs api\\
The web ui is built on the api and can be used with the same query parameters as the api. On top of that the web ui has the same limitations as the api, except possible rate-limiting. Further on the api will be used as it allows to query quicker and with less overhead.\\

directly search for filename\\
search for filename in the description or readme of any repository\\

current limitations:\\
Per search query a maximum result of 1000 entries can be returned.\\
Meaning a naive search for docker-compose would result in a total\_count of 800k results of which you can only view 1000.\\

Ways to mitigate that:\\
GitHub offers various kinds of query parameters\\
<<possible list some and explain>>\\
Querying the api allows only the following parameters, where q is further divided into extra "parameters":\\
q=\{extension\\
file size\\
path\\
file name\}\\
sort\\
order\\
page\\
\\
to get unique urls and to be able to search for more than just 1000 files, we will utilize the size parameter, which allows values of either Integer or Integer..Integer, meaning a range. The values supplied will be interpreted by GitHub as bytes. Therefore we can search for more files as on a byte level files are more likely to be distributed as not all share the same size due to obvious reasons (content). <<refer to how docker-compose files are structured>>

\section{Database}
The aim is to build a knowledge graph or graph in general to be able to utilize it further for either recommender systems or proper analysis of relations between the usage of images.

Comparisson between possible Databases?\\

Neo4j\\
- proper Graph Database\\
- querying feels super outdated\\
- only directed edges\\
- node.js support\\

Grakn.ai\\
- underlying system is a Graph, but more knowledge orientated\\
- designed for machines (output/input)\\
- hypergraph\\
- node.js support\\
\\
Currently in favour of grakn.\\
Next step: define Modell\\

\input{bib.tex}

\end{document}