\chapter{Evaluation}

Second most important chapter. Verifies the theses defined in the previous chapter. Tries to evaluate and analyze the contribution in qualitative or quantitative terms. Ends with a discussion. Approximately 20 to 30 pages. Can be split into multiple chapters.

% Hier solltest du die limitations deiner Contirbution versuchen zu analysieren. Dies können quantitative sachen sein, wie die limitationen des crawlers wie oft gecrawlet werden kann über github api und wie oft dann andere ergebnisse rauskommen. andere sachen wären zB ob der von dir erstellte score "qualitativ gut" IaC bewerten kann. Für diese unterschiedlichen Punkte sollten folgende Unterpunkte folgen:
%a) Evaluation Setup: hier wird der versuchsaufbau und die durchführung beschrieben. welche größen werden zB gemessen sollte auch beschrieben werden. Womit wird vielleicht das ergebnis verglichen?
%b) Results: Darstellung der Ergebnisse und Beschreibung der Ergebnisse
%c) Discussion: Bewerte die Ergebnisse und ordnete sie ein.
%Am Ende des Kapitels sollte dann eine summary kommen, was der leser gelernt haben soll und welche fragen noch offen geblieben sind für future work und welche forschungsrichtungen man diese arbeit weiter entwickeln kann.

\section{Graph Analysis}
\section{TBD - General Analysis}