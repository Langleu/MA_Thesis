\documentclass[12pt,a4paper,twoside]{book}

%% general usepackage stuff
\usepackage{times}
\usepackage{mathptmx}
\usepackage{booktabs}
\usepackage{graphicx}
\usepackage{titlesec}
\usepackage{amssymb}

\usepackage[main=english,ngerman]{babel}
\usepackage[T1]{fontenc}

\usepackage{color}

%% labels and page stuff
\renewcommand{\labelitemi}{$\diamond$}
\renewcommand{\labelitemii}{$\circ$}

\setcounter{tocdepth}{4} %% TODO: remove later again
\setcounter{secnumdepth}{3}

%\setlength{\oddsidemargin}{-0.5cm}
%\setlength{\evensidemargin}{-0.5cm}

\setlength{\oddsidemargin}{4.58mm}%final
\setlength{\evensidemargin}{-4.92mm}%final


\setlength{\textwidth}{15.95cm}%final
\setlength{\textheight}{23.5cm}
\setlength{\voffset}{-1.45cm} %


%% todo
\newcommand{\todo}[1]{\textcolor{red}{[todo: }#1\textcolor{red}{]}}
\newcommand{\todoh}[1]{} 
\newcommand{\done}[1]{} 

%% smallerText
\newcommand{\smallerTextSize}{10}
\newcommand{\smallerTextSkip}{12}
\newcommand{\smallerBegin}{\fontsize{\smallerTextSize}{\smallerTextSkip}\selectfont}
\newcommand{\smallerEnd}{\normalsize}
\newcommand{\smaller}[1]{\smallerBegin #1\smallerEnd}

%%reference to defintion
\newcommand{\defref}[1]{\ref{#1} on page \pageref{#1}}

%% special, short for shell
\newcommand{\shl}[1]{\sf \small #1\rm\normalsize}

%% indentation in tables
\newcommand{\tabind}[1]{\rule{#1mm}{0cm}}

%% smaller in captions
\newcommand{\captionfonts}{\smallerBegin}
                                                                                                                                                                              
\makeatletter  % Allow the use of @ in command names
\long\def\@makecaption#1#2{%
  \vskip\abovecaptionskip
  \sbox\@tempboxa{{\captionfonts #1: #2}}%
  \ifdim \wd\@tempboxa >\hsize
    {\captionfonts #1: #2\par}
  \else
    \hbox to\hsize{\hfil\box\@tempboxa\hfil}%
  \fi
  \vskip\belowcaptionskip}
\makeatother   % Cancel the effect of \makeatletter

%% clear page before new chapter
\makeatletter
\def\cleardoublepage{\clearpage\if@twoside \ifodd\c@page\else
\hbox{}
\vspace*{\fill}
\begin{center}
%This page intentionally contains only this sentence.
\end{center}
\vspace{\fill}
\thispagestyle{empty}
\newpage
\if@twocolumn\hbox{}\newpage\fi\fi\fi}
\makeatother

%% abbreviations
\usepackage{nomencl}
\let\abbrev\nomenclature
\renewcommand{\nomname}{List of Abbreviations}
\setlength{\nomlabelwidth}{.25\hsize}
\renewcommand{\nomlabel}[1]{#1 \dotfill}
\setlength{\nomitemsep}{-\parsep}
\makenomenclature
\newcommand{\Listofabbrev}{
\printnomenclature
\newpage
}

%% chapter title formatting
\titleformat{\chapter}[display]{ \raggedleft }{\fontsize{52}{63}\selectfont \bf \thechapter }{0.2cm}{\fontsize{32}{38.7}\selectfont  }[]

%% haeder formatting
\renewcommand{\chaptermark}[1]{%
\markboth{\chaptername
\ \thechapter.\ #1}{}}
\renewcommand{\sectionmark}[1]{\markright{\thesection.\ #1}}
\usepackage{fancyhdr}
\pagestyle{fancy}
\fancyhf[LEH,ROH]{\thepage}
\fancyhf[REH]{\smaller{\nouppercase{\leftmark}}}
\fancyhf[LOH]{\smaller{\it \nouppercase{\rightmark}}}
\fancyhf[COF]{\rule{0.2cm}{0.0cm}}
\fancyhf[CEF]{\rule{0.2cm}{0.0cm}}
\renewcommand{\headrulewidth}{0pt}

%% macro for figures
%\usepackage{svg}
%\usepackage{amsmath}
%\newcommand{\printlabel}{}
%\newcommand{\abcdef}[1]{\tiny #1 \normalsize}

		%% arguments: graphics file, label, caption, smallcaption
\newcommand{\insertFigure}[4]{\begin{figure}[top] \smallerBegin \centering \includegraphics{#1}  \\  \caption{\label{#2}\smallerBegin #3 \footnotesize{#4}  \smallerEnd }  
\end{figure}	}
%% macro for figures with short caption
		%% arguments: graphics file, label, caption, smallcaption, shortcaption
\newcommand{\insertFigureShort}[5]{\begin{figure}[top] \smallerBegin \centering \includegraphics{#1} \label{#2} \\ \caption[#5]{\smallerBegin #3 \footnotesize{#4} \smallerEnd } 
\end{figure}	}
%\renewcommand{\topfraction}{1}

\newcommand{\spaceafterpar}{\vspace{14.48pt}}

\renewcommand{\floatpagefraction}{.75} % vorher: .5
\renewcommand{\textfraction}{.1}       % vorher: .2
\renewcommand{\topfraction}{.8}        % vorher: .7
\renewcommand{\bottomfraction}{.5} 
\setcounter{topnumber}{3}              % vorher: 2
\setcounter{bottomnumber}{2}           % vorher: 1
\setcounter{totalnumber}{5}            % vorher: 3
%from: http://www.matthiaspospiech.de/latex/vorlagen/allgemein/preambel/9/

%macros for tables
    %% arguments: columns
\newcommand{\tableBegin}[1]{\begin{table}[top] \begin{center} \smallerBegin \begin{tabular}{#1}}
		%% arguemtns: caption, label
\newcommand{\tableEnd}[2]{ \end{tabular} \smallerEnd \end{center} \caption{#1} \label{#2}	\end{table}}

%% shortcuts
\newcommand{\emit}[1]{\item \emph{#1}:}
\newcommand{\firstappear}[2]{\emph{#1} (#2) \abbrev{#2}{#1}}

\hyphenation{or-gan-izing}

%% definitions

\newtheorem{definition}{Definition}

%% examples
\newcommand{\exampleBeginText}[1]{\paragraph{#1}}
\newcommand{\exampleEnd}{\vspace{6mm}}
\newcommand{\exampleBegin}{\exampleBeginText{Example}}

%algorithms
\usepackage[english,ruled,vlined, slide, norelsize]{algorithm2e}

%placements
\usepackage{float}

%references
\usepackage[numbers,sort]{natbib}
\setlength{\bibsep}{0.0pt}
\bibliographystyle{plainnat}

%ref imgs
\usepackage{hyperref}
\usepackage{cleveref}
\usepackage{todonotes}
\usepackage[onehalfspacing]{setspace}
\usepackage{tikz}
\usetikzlibrary{spy}
\usepackage{listings}
\usepackage{color}

\usepackage{csvsimple}